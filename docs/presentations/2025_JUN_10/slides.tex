% LTeX: enabled=false
\documentclass[letterpaper,10pt,aspectratio=169,dvipsnames]{beamer}

\usetheme{mcgill}
\usefonttheme{serif}

\usepackage[nolist,nohyperlinks]{acronym}
\begin{acronym}
  \acro{CBV}{Call-By-Value}
  \acro{CBN}{Call-By-Name}
  \acro{CBPV}{Call-By-Push-Value}
\end{acronym}
\usepackage{amsmath}
\usepackage{amssymb}
\usepackage{amsthm}
\usepackage{graphicx}
\usepackage{listings}
\usepackage{mathpartir}
\usepackage{noto-mono}
\usepackage{noto-sans}
\usepackage{noto-serif}
\usepackage{stmaryrd}
\usepackage{tikz}
\usetikzlibrary{arrows,calc,matrix}
\pgfkeys%
{%
  /tikz/fulloverlay/.code=%
  {%
    \pgfkeysalso%
    {%
      overlay,%
      remember picture,%
      shift=(current page.south west),%
      x=(current page.south east),%
      y=(current page.north west)%
    }%
  }%
}
\tikzset%
{%
  invisible/.style={opacity=0,text opacity=0},%
  visible on/.style={alt={#1{}{invisible}}},%
  alt/.code args=%
  {<#1>#2#3}%
  {%
    \alt<#1>{\pgfkeysalso{#2}}{\pgfkeysalso{#3}} % \pgfkeysalso doesn't change the path
  },%
}
\usepackage{xspace}

% Helper macros
\NewDocumentCommand{\binderDot}{}%
{\ensuremath{\mathbin{.}}}

% Type symbols
\NewDocumentCommand{\tyIntSymb}{}%
{{\ensuremath{\mathtt{Int}}}}
\NewDocumentCommand{\tyBoolSymb}{}%
{{\ensuremath{\mathtt{Bool}}}}
\NewDocumentCommand{\tyUpSymb}{}%
{{\ensuremath{\mathtt{\uparrow}}}}
\NewDocumentCommand{\tyDownSymb}{}%
{{\ensuremath{\mathtt{\downarrow}}}}
\NewDocumentCommand{\tyFunSymb}{}%
{{\ensuremath{\mathtt{\to}}}}

% Types
\NewDocumentCommand{\tyInt}{}%
{{\ensuremath{\tyIntSymb}}}
\NewDocumentCommand{\tyBool}{}%
{{\ensuremath{\tyBoolSymb}}}
\NewDocumentCommand{\tyUp}{m}%
{{\ensuremath{#1 \tyUpSymb}}}
\NewDocumentCommand{\tyDown}{m}%
{{\ensuremath{#1 \tyDownSymb}}}
\NewDocumentCommand{\tyFun}{m m}%
{{\ensuremath{#1 \tyFunSymb #2}}}

% Term symbols
\NewDocumentCommand{\tmTrueSymb}{}%
{{\ensuremath{\mathtt{true}}}}
\NewDocumentCommand{\tmFalseSymb}{}%
{{\ensuremath{\mathtt{false}}}}
\NewDocumentCommand{\tmThunkSymb}{}%
{{\ensuremath{\mathtt{thunk}}}}
\NewDocumentCommand{\tmForceSymb}{}%
{{\ensuremath{\mathtt{force}}}}
\NewDocumentCommand{\tmReturnSymb}{}%
{{\ensuremath{\mathtt{return}}}}
\NewDocumentCommand{\tmToSymb}{}%
{{\ensuremath{\mathtt{to}}}}
\NewDocumentCommand{\tmPrintIntSymb}{}%
{{\ensuremath{\mathtt{printInt}}}}
\NewDocumentCommand{\tmThenSymb}{}%
{{\ensuremath{\mathtt{then}}}}

% Terms
\NewDocumentCommand{\tmTrue}{}%
{{\ensuremath{\tmTrueSymb}}}
\NewDocumentCommand{\tmFalse}{}%
{{\ensuremath{\tmFalseSymb}}}
\NewDocumentCommand{\tmThunk}{o m}%
{{\ensuremath{{\IfValueT{#1}{#1}\tmThunkSymb}(#2)}}}
\NewDocumentCommand{\tmForce}{m}%
{{\ensuremath{\tmForceSymb(#1)}}}
\NewDocumentCommand{\tmReturn}{o m}%
{{\ensuremath{{\IfValueT{#1}{#1}\tmReturnSymb}(#2)}}}
\NewDocumentCommand{\tmTo}{m m m}%
{{\ensuremath{#1\ \tmToSymb\ #2 \binderDot #3}}}
\NewDocumentCommand{\tmLam}{m m}%
{{\ensuremath{\lambda #1 \binderDot #2}}}
\NewDocumentCommand{\tmApp}{m m}%
{{\ensuremath{#1 \mathbin{ } #2}}}
\NewDocumentCommand{\tmPrintInt}{m m}%
{{\ensuremath{\tmPrintIntSymb\ #1\ \tmThenSymb\ #2}}}

\NewDocumentCommand{\vtyp}{O{\Gamma} m m}%
{{\ensuremath{#1 \vdash^v #2 : #3}}}

\NewDocumentCommand{\ctyp}{O{\Gamma} m m}%
{{\ensuremath{#1 \vdash^c #2 : #3}}}

\newcommand{\lambdacomp}{\texorpdfstring{\(\lambda\)-comp}{lambdacomp}\xspace}
\newcommand{\lambdacalcplus}%
{%
  \texorpdfstring%
  {\(\lambda\)-Calculus (+ \(\alpha\))}%
  {Lambda Calculus (+ more)}%
  \xspace%
}
\newcommand{\JJ}{Junyoung~Jang}
\newcommand{\JJemail}{{\scriptsize junyoung.jang@mail.mcgill.ca}}
\newcommand{\JJuniv}{{\scriptsize McGill University}}

\title%
[\lambdacomp: A Primitive \lambdacalcplus Compiler Based on CBPV]%
{\lambdacomp}%
\subtitle{A Primitive \lambdacalcplus Compiler Based on CBPV}%
\author[\JJ]{%
  \texorpdfstring%
  {\parbox{10em}{\centering \JJ\\\JJemail\\\JJuniv}}%
  {\JJ}%
}%
\date{}%

% LTeX: enabled=true

\begin{document}
\begin{frame}[plain]
  \titlepage
\end{frame}

\begin{frame}{Motivation}
  \begin{tikzpicture}[fulloverlay]
    \node at (0.5, 0.8) {Why \textcolor<2-3>{orange}{compiling} \textcolor<4->{orange}{\acf{CBPV}?}};

    \node[visible on=<3>,text width=\textwidth,align=center] at (0.5, 0.5)
    {%
      To understand how to compile modal \(\lambda\)-calculus\\
      (as an experience)%
    };

    \node[visible on=<5->] at (0.5, 0.55)
    {Being closer to low-level (value-computation distinction)};

    \node[visible on=<6->] at (0.5, 0.45)
    {Allowing simple yet useful optimizations};
  \end{tikzpicture}
\end{frame}

\begin{frame}[fragile]{\acf{CBPV}}
  \begin{tikzpicture}[fulloverlay]
    \node[visible on=<2->,alt=<9-11>{gray}{},alt=<10>{orange}{},alt=<12>{orange}{},alt=<14->{gray}{}] (vttitle) at (0.3, 0.8)
    {Value Types\qquad\(A,B,\ldots\)};

    \node[visible on=<3->,alt=<4-7>{gray}{},alt=<8>{orange}{},alt=<14->{gray}{}] (cttitle) at (0.7, 0.8)
    {Computation Types\qquad\(S,T,\ldots\)};

    \matrix[visible on=<4->,alt=<9-12>{gray}{},alt=<14->{gray}{},align=center,matrix of nodes,%
    row 1/.style={visible on=<5->,alt=<5>{orange}{}},%
    row 2/.style={visible on=<6->,alt=<6>{orange}{}},%
    row 3/.style={visible on=<7->,alt=<7-8>{orange}{}}] (vtdesc) at (0.3, 0.7)
    {%
      \tyInt\\
      \tyBool\\
      \tyUp{S}\\
    };

    \matrix[visible on=<8->,alt=<14->{gray}{},align=center,matrix of nodes,%
    row 1/.style={visible on=<9->,alt=<9-10>{orange}{}},%
    row 2/.style={visible on=<11->,alt=<11-12>{orange}{}}] (ctdesc) at (0.7, 0.7)
    {%
      \tyDown{A}\\
      \tyFun{A}{S}\\
    };

    \draw[visible on=<8>,<-,orange] (vtdesc-3-1)
    -| node[midway,below] {\scriptsize as a deferred computation}
    ($(cttitle.west) - (0.05, 0)$)
    -- (cttitle);

    \draw[visible on=<10>,<-,orange] (ctdesc-1-1)
    -| node[midway,below] {\scriptsize as a return value}
    ($(vttitle.east) + (0.05, 0)$)
    -- (vttitle);

    \draw[visible on=<12>,<-,orange] (ctdesc-2-1)
    -| node[midway,below] {\scriptsize as an argument value}
    ($(vttitle.east) + (0.05, 0)$)
    -- (vttitle);

    \node[visible on=<14->,alt=<23-28>{gray}{},alt=<30-31>{orange}{}] (vtitle) at (0.3, 0.5)
    {Values \qquad\(U,V,W,\ldots\)};

    \node[visible on=<15->,alt=<16-21>{gray}{},alt=<22>{orange}{}] (ctitle) at (0.7, 0.5)
    {Computations\qquad\(L,M,N,\ldots\)};

    \matrix[visible on=<16->,alt=<23->{gray}{},align=center,matrix of nodes,%
    row 1/.style={visible on=<17->,alt=<17>{orange}{}},%
    row 2/.style={visible on=<18->,alt=<18>{orange}{}},%
    row 3/.style={visible on=<19->,alt=<19>{orange}{}},%
    row 4/.style={visible on=<20->,alt=<20>{orange}{}},%
    row 5/.style={visible on=<21->,alt=<21-22>{orange}{},alt=<24>{orange}{}}] (vdesc) at (0.3, 0.3)
    {%
      \(x\) & {\small (Variable)}\\
      \(i\) & {\small (Integer)}\\
      \tmTrue & \\
      \tmFalse & \\
      \tmThunk{L} & \\
    };

    \draw[visible on=<22>,<-,orange] (vdesc-5-1) -| ($(ctitle.west) - (0.05, 0)$) -- (ctitle);

    \matrix[visible on=<23->,align=center,matrix of nodes,%
    row 1/.style={visible on=<24->,alt=<24>{orange}{},alt=<30>{orange}{}},%
    row 2/.style={visible on=<25->,alt=<25>{orange}{},alt=<30-31>{orange}{}},%
    row 3/.style={visible on=<26->,alt=<26>{orange}{}},%
    row 4/.style={visible on=<27->,alt=<27>{orange}{}},%
    row 5/.style={visible on=<28->,alt=<28>{orange}{},alt=<30-31>{orange}{}},%
    row 6/.style={visible on=<29->,alt=<29>{orange}{}}] (cdesc) at (0.7, 0.3)
    {%
      \tmForce{U}\\
      \tmReturn{U}\\
      \tmTo{L}{x}{M}\\
      \tmLam{x}{L}\\
      \tmApp{L}{U}\\
      \tmPrintInt{U}{L}\\
    };

    \draw[visible on=<24>,<->,orange] (vdesc-5-1) -| (cdesc-1-1);

    \draw[visible on=<30>,<-,orange] (cdesc-1-1) -| ($(vtitle.east) + (0.05, 0)$) -- (vtitle);
    \begin{scope}[every path/.style={visible on=<30-31>,<-,orange}]
      \draw (cdesc-2-1) -| ($(vtitle.east) + (0.05, 0)$) -- (vtitle);
      \draw (cdesc-5-1) -| ($(vtitle.east) + (0.05, 0)$) -- (vtitle);
    \end{scope}
  \end{tikzpicture}
\end{frame}

\begin{frame}{Usual \(\lambda\)-Calculus Compile}
  \begin{tikzpicture}[fulloverlay]
    \begin{scope}[every node/.style={draw,rectangle,alt=<2>{gray}{}}]
      \node (steps-1) at (0.5, 0.7) {\strut Elaboration/Type Checking};
      \node (steps-3) at (0.5, 0.3) {\strut Code Generation};
    \end{scope}
    \matrix[draw,rectangle,matrix of nodes,row sep={0.1\paperheight,between origins},nodes={draw,rectangle},alt=<2>{orange}{}] (steps-2) at (0.5, 0.5)
    {
      \node at (0.1,0) {\strut Arity Analysis};
      \node at (-0.1,0) {\strut Closure Conversion};\\
      \strut Optimizations\\
    };
    \foreach \a in {1,2}%
    {%
      \pgfmathtruncatemacro\b{\a+1}
      \draw[alt=<2>{gray}{},->] (steps-\a) -- (steps-\b);%
    }
    % \foreach \a in {2,3}%
    % {%
    %   \pgfmathtruncatemacro\b{\a+1}
    %   \draw[->] (steps-\a-1) -- (steps-\b-1);%
    % }
  \end{tikzpicture}
\end{frame}

\begin{frame}{\(\lambda\)-Abstraction as Computation}
  \begin{tikzpicture}[fulloverlay]
    \node[visible on=<-2>] at (0.5,0.5)
    {\(\tmApp {(\tmLam x {\tmReturn x})} {(\tmLam x {\tmReturn x})}\)};

    \begin{scope}[visible on=<2>,red]
      \draw (0.3, 0.6) -- (0.7, 0.4);
      \draw (0.7, 0.6) -- (0.3, 0.4);
    \end{scope}

    \node[visible on=<3->] at (0.5,0.5)
    {\(\tmApp {(\tmLam x {\tmReturn x})} {(\tmThunk {\tmLam x {\tmReturn x}})}\)};

    \node[visible on=<4>,orange] at (0.5,0.35)
    {We never move \((\tmLam x M)\) around solely!};
  \end{tikzpicture}
\end{frame}

\begin{frame}{No Closure?}
  \begin{tikzpicture}[fulloverlay]
    \node at (0.25, 0.7) {Usual \(\lambda\)-Calculus};

    \node[visible on=<2->] at (0.25, 0.5) {\small \((\tmLam x t)\) can be a value};
    \node[visible on=<3->,text width=16em,align=center] at (0.25, 0.4)
    {\small \(\Longrightarrow\) \((\tmLam x t)\) can be a closure\\with its free variables};

    \node[visible on=<4->] at (0.75, 0.7) {CBPV};

    \node[visible on=<5->] at (0.75, 0.5) {\small \((\tmLam x L)\) is not a value};
    \node[visible on=<6->,alt=<7>{red}{}] at (0.75, 0.4)
    {\small \(\Longrightarrow\) there is no closure{\color<-6>{white}\(\ldots\)?}};
  \end{tikzpicture}
\end{frame}

\begin{frame}{Where Has the Closure Gone?}
  \begin{tikzpicture}[fulloverlay]
    \begin{scope}[visible on=<-3>]
      \begin{scope}[alt=<2->{gray}{}]
        \node (vttitle) at (0.3, 0.8)
        {Value Types\qquad\(A,B,\ldots\)};

        \node (cttitle) at (0.7, 0.8)
        {Computation Types\qquad\(S,T,\ldots\)};

        \matrix[align=center,matrix of nodes] (vtdesc) at (0.3, 0.7)
        {%
          \tyInt\\
          \tyBool\\
          \tyUp{S}\\
        };

        \matrix[align=center,matrix of nodes] (ctdesc) at (0.7, 0.7)
        {%
          \tyDown{A}\\
          \tyFun{A}{S}\\
        };

        \node (ctitle) at (0.7, 0.5)
        {Computations\qquad\(L,M,N,\ldots\)};

        \matrix[align=center,matrix of nodes] (cdesc) at (0.7, 0.3)
        {%
          \tmForce{U}\\
          \tmReturn{U}\\
          \tmTo{L}{x}{M}\\
          \tmLam{x}{L}\\
          \tmApp{L}{U}\\
          \tmPrintInt{U}{L}\\
        };
      \end{scope}

      \begin{scope}[alt=<2>{orange}{}]
        \node (vtitle) at (0.3, 0.5)
        {Values \qquad\(U,V,W,\ldots\)};

        \matrix[align=center,matrix of nodes,%
        row 5/.style={alt=<3>{orange}{}}] (vdesc) at (0.3, 0.3)
        {%
          \(x\)\\
          \(i\)\\
          \tmTrue\\
          \tmFalse\\
          \tmThunk{L}\\
        };
      \end{scope}
    \end{scope}

    \begin{scope}[visible on=<4->]
      \node at (0.25, 0.7) {Usual \(\lambda\)-Calculus};

      \node at (0.25, 0.5) {\small \((\tmLam x e)\) can be a value};
      \node[text width=16em,align=center] at (0.25, 0.4)
      {\small \(\Longrightarrow\) \((\tmLam x e)\) can be a closure\\with its free variables};

      \node at (0.75, 0.7) {CBPV};

      \node[visible on=<5->,orange] at (0.75, 0.5) {\small \((\tmThunk{L})\) is a value};
      \node[visible on=<6->,orange,text width=16em,align=center] at (0.75, 0.4)
      {\small \(\Longrightarrow\) \((\tmThunk{L})\) is a closure\\with its free variables};
    \end{scope}
  \end{tikzpicture}
\end{frame}

\begin{frame}{Closures in a General Modal Calculus}
  \begin{tikzpicture}[fulloverlay]
    \node at (0.5, 0.7) {General Modal Calculus};

    \node[visible on=<2->] at (0.5, 0.5)
    {\small Language construct \(C\) can be a value
      {\color<-3>{white}with some computation}};

    \node[visible on=<3-4>] at (0.5, 0.4)
    {\small \(\Longrightarrow\) \(C\) can be a closure with its free variables};
  \end{tikzpicture}
\end{frame}

\begin{frame}{Arity Analysis}
  \begin{tikzpicture}[fulloverlay]
    \node[visible on=<1>] at (0.5, 0.5) {Exactly when is \((\tmLam x e)\) a value?};

    \node[visible on=<2->,text width=0.8\textwidth,align=center] at (0.5, 0.5)
    {How many arguments are applied to a given function definition\\
      \(f {=} (\tmLam {x_1} {\tmLam {x_2} {\ldots \tmLam {x_n} e}})\)
      at least when it is called?};

    \node[visible on=<3->,text width=\textwidth,align=center] at (0.5, 0.3)
    {Less closures\\(less calls on allocation functions, less runtime fragmentations, \ldots)};
  \end{tikzpicture}
\end{frame}

\begin{frame}{How to Represent Arity?}
  \begin{tikzpicture}[fulloverlay]
    \node at (0.5, 0.6)
    {Suppose that \(f {=} (\tmLam {x_1} {\tmLam {x_2} {\ldots \tmLam {x_n} e}})\)
      takes at least 2 arguments in calling sites};

    \node[visible on=<2->] at (0.5, 0.5)
    {So?};

    \node[visible on=<3->,orange] at (0.5, 0.4)
    {No way to represent this in \(\lambda\)-calculus without an IR (CPS, A-normal form, \ldots)};
  \end{tikzpicture}
\end{frame}

\begin{frame}{Arity Analysis for \ac{CBPV}}
  \begin{tikzpicture}[fulloverlay]
    \node[visible on=<1-2>,text width=\textwidth,align=center] at (0.5, 0.55)
    {\(f {=} (\tmReturn{\tmThunk{\tmLam {x_1} {\tmReturn[\color<2>{orange}]{\tmThunk[\color<2>{orange}]{\tmLam {x_2} {(\ldots \tmReturn{\tmThunk{\tmLam {x_n} L}})}}}}}})\)\\
      taking 2 arguments};
    \node[visible on=<3->,text width=\textwidth,align=center] at (0.5, 0.55)
    {\(f {=} (\tmReturn{\tmThunk{\tmLam {x_1} {\tmLam {x_2} {(\ldots \tmReturn{\tmThunk{\tmLam {x_n} L}})}}}})\)\\
      taking 2 arguments};

    \node[visible on=<4->,orange] at (0.5, 0.45)
    {No closure for the first argument application!};

    \node[visible on=<5>,text width=\textwidth,align=center] at (0.5, 0.3)
    {Previous call for \(f\)\\
      \(\tmTo f {f_1} {(\tmTo {\tmApp {\tmForce{f_1}} {a_1}} {f_2} {(\tmTo {\tmApp {\tmForce{f_2}} {a_2}} {f_3} L)})}\)};

    \node[visible on=<6>,text width=\textwidth,align=center] at (0.5, 0.3)
    {New call for \(f\)\\
      \(\tmTo f {f_1} {(\tmTo {\tmApp {(\tmApp {\tmForce{f_1}} {a_1})} {a_2}} {f_3} L)}\)};
  \end{tikzpicture}
\end{frame}

% \begin{frame}{How to Apply Arity Analysis?}
%   \begin{tikzpicture}[fulloverlay]
%     \node[alt=<2->{gray}{}] at (0.5, 0.75)
%     {\(f {=} (\tmReturn{\tmThunk{\tmLam {x_1} {\tmReturn{\tmThunk{\tmLam {x_2} {(\ldots \tmReturn{\tmThunk{\tmLam {x_n} L}})}}}}}})\)};
%     \node[alt=<2->{gray}{}] at (0.5, 0.65)
%     {\(\Longrightarrow f {=} (\tmReturn{\tmThunk{\tmLam {x_1} {\tmLam {x_2} {(\ldots \tmReturn{\tmThunk{\tmLam {x_n} L}})}}}})\)};

%     \node at (0.5, 0.5)
%     {\(\tmTo f {{\color<2>{orange}f_1}} {(\tmTo {\tmApp {\tmForce{{\color<2>{orange}f_1}}} {a_1}} {f_2} {(\tmTo {\tmApp {\tmForce{f_2}} {a_2}} {f_3} L)})}\)};
%     \node at (0.5, 0.4)
%     {\(\Longrightarrow \tmTo f {f_1} {(\tmTo {\tmApp {(\tmApp {\tmForce{f_1}} {a_1})} {a_2}} {f_3} L)}\)};

%     \node[visible on=<2->] at (0.5, 0.3)
%     {What if \({\color{orange}f_1}\) is passed arround?};
%   \end{tikzpicture}
% \end{frame}

\begin{frame}{\(\eta\)-Reducing Optimization}
  \begin{tikzpicture}[fulloverlay]
    \node[visible on=<1->] at (0.5, 0.65)
    {In (effectful) \(\lambda\)-Calculus: \((\tmLam x {\tmApp e x}) \not\equiv_{\mathtt{obs}} e\)};

    \node[visible on=<2->] at (0.5, 0.55)
    {In (effectful) CBPV: \((\tmLam x {\tmApp L x}) \equiv_{\mathtt{obs}} L\)};

    \node[visible on=<3->] at (0.5, 0.45)
    {In (effectful) CBPV: \((\tmThunk{\tmForce V}) \equiv_{\mathtt{obs}} V\)};

    \node[visible on=<4->] at (0.5, 0.3)
    {In (effectful) CBPV: all negative types satisfy \(\eta\)-equivalence};
  \end{tikzpicture}
\end{frame}

\begin{frame}{Optimizations in General Modal Calculus}
  \begin{tikzpicture}[fulloverlay]
    \node at (0.5, 0.7) {General Modal Calculus};

    \node[visible on=<2->] at (0.5, 0.5)
    {\small Modality plays a key role in both representing
      optimization results and justifying optimizations.};
  \end{tikzpicture}
\end{frame}

\begin{frame}{Demo}
  \begin{tikzpicture}[fulloverlay]
    \node at (0.5, 0.5) {Let's try some examples!};
  \end{tikzpicture}
\end{frame}

\begin{frame}{Take Away}
  \pause
  \begin{itemize}[<+->]
  \item CBPV uses \(\tmThunk{L}\) as a value containing computation\\
    \(\Longrightarrow\) \(\tmThunk{L}\) is the closure in CBPV
  \item CBPV exposes \(\eta\)-equivalence for negative types using its modalities\\
    \(\Longrightarrow\) \(\eta\)-reduction is a safe optimization\\[2em]
  \item Compile of a general modal calculus should also consider these two:\\
    \begin{itemize}
    \item \vspace{-0.5em}What are the values with computation?
    \item \vspace{-0.75em}What equivalences are allowed by the modalities?
    \end{itemize}
  \end{itemize}
\end{frame}

\begin{frame}{Typing Rules of CBPV}
  \begin{mathpar}
    \inferrule
    {\ctyp L S}
    {\vtyp {\tmThunk L} {\tyUp S}}

    \inferrule
    {\vtyp U {\tyUp S}}
    {\ctyp {\tmForce U} S}
    \\

    \inferrule
    {\vtyp U A}
    {\ctyp {\tmReturn U} {\tyDown A}}

    \inferrule
    {\ctyp U {\tyDown A}\\
      \ctyp [\Gamma, x{:}A] U B}
    {\ctyp {\tmTo L x M} B}
  \end{mathpar}
\end{frame}
\end{document}
