% LTeX: enabled=false
\documentclass[letterpaper,10pt,aspectratio=169,dvipsnames]{beamer}

\usetheme{mcgill}
\usefonttheme{serif}

\usepackage[nolist,nohyperlinks]{acronym}
\begin{acronym}
  \acro{CBV}{Call-By-Value}
  \acro{CBN}{Call-By-Name}
  \acro{CBPV}{Call-By-Push-Value}
\end{acronym}
\usepackage{amsmath}
\usepackage{amssymb}
\usepackage{amsthm}
\usepackage{graphicx}
\usepackage{listings}
\usepackage{mathpartir}
\usepackage{noto-mono}
\usepackage{noto-sans}
\usepackage{noto-serif}
\usepackage{stmaryrd}
\usepackage{tikz}
\usetikzlibrary{arrows,calc,matrix}
\pgfkeys%
{%
  /tikz/fulloverlay/.code=%
  {%
    \pgfkeysalso%
    {%
      overlay,%
      remember picture,%
      shift=(current page.south west),%
      x=(current page.south east),%
      y=(current page.north west)%
    }%
  }%
}
\tikzset%
{%
  invisible/.style={opacity=0,text opacity=0},%
  visible on/.style={alt={#1{}{invisible}}},%
  alt/.code args=%
  {<#1>#2#3}%
  {%
    \alt<#1>{\pgfkeysalso{#2}}{\pgfkeysalso{#3}} % \pgfkeysalso doesn't change the path
  },%
}
\usepackage{xspace}

% Helper macros
\NewDocumentCommand{\binderDot}{}%
{\ensuremath{\mathbin{.}}}

% Type symbols
\NewDocumentCommand{\tyIntSymb}{}%
{{\ensuremath{\mathtt{Int}}}}
\NewDocumentCommand{\tyBoolSymb}{}%
{{\ensuremath{\mathtt{Bool}}}}
\NewDocumentCommand{\tyUpSymb}{}%
{{\ensuremath{\mathtt{\uparrow}}}}
\NewDocumentCommand{\tyDownSymb}{}%
{{\ensuremath{\mathtt{\downarrow}}}}
\NewDocumentCommand{\tyFunSymb}{}%
{{\ensuremath{\mathtt{\to}}}}

% Types
\NewDocumentCommand{\tyInt}{}%
{{\ensuremath{\tyIntSymb}}}
\NewDocumentCommand{\tyBool}{}%
{{\ensuremath{\tyBoolSymb}}}
\NewDocumentCommand{\tyUp}{m}%
{{\ensuremath{#1 \tyUpSymb}}}
\NewDocumentCommand{\tyDown}{m}%
{{\ensuremath{#1 \tyDownSymb}}}
\NewDocumentCommand{\tyFun}{m m}%
{{\ensuremath{#1 \tyFunSymb #2}}}

% Term symbols
\NewDocumentCommand{\tmTrueSymb}{}%
{{\ensuremath{\mathtt{true}}}}
\NewDocumentCommand{\tmFalseSymb}{}%
{{\ensuremath{\mathtt{false}}}}
\NewDocumentCommand{\tmThunkSymb}{}%
{{\ensuremath{\mathtt{thunk}}}}
\NewDocumentCommand{\tmForceSymb}{}%
{{\ensuremath{\mathtt{force}}}}
\NewDocumentCommand{\tmReturnSymb}{}%
{{\ensuremath{\mathtt{return}}}}
\NewDocumentCommand{\tmToSymb}{}%
{{\ensuremath{\mathtt{to}}}}
\NewDocumentCommand{\tmPrintIntSymb}{}%
{{\ensuremath{\mathtt{printInt}}}}
\NewDocumentCommand{\tmBeforeSymb}{}%
{{\ensuremath{\mathtt{before}}}}

% Terms
\NewDocumentCommand{\tmTrue}{}%
{{\ensuremath{\tmTrueSymb}}}
\NewDocumentCommand{\tmFalse}{}%
{{\ensuremath{\tmFalseSymb}}}
\NewDocumentCommand{\tmThunk}{m}%
{{\ensuremath{\tmThunkSymb(#1)}}}
\NewDocumentCommand{\tmForce}{m}%
{{\ensuremath{\tmForceSymb(#1)}}}
\NewDocumentCommand{\tmReturn}{m}%
{{\ensuremath{\tmReturnSymb(#1)}}}
\NewDocumentCommand{\tmTo}{m m m}%
{{\ensuremath{#1\ \tmToSymb\ #2 \binderDot #3}}}
\NewDocumentCommand{\tmLam}{m m}%
{{\ensuremath{\lambda #1 \binderDot #2}}}
\NewDocumentCommand{\tmApp}{m m}%
{{\ensuremath{#1 \mathbin{ } #2}}}
\NewDocumentCommand{\tmPrintInt}{m m}%
{{\ensuremath{\tmPrintIntSymb #1 \tmBeforeSymb #2}}}

\newcommand{\lambdacomp}{\texorpdfstring{\(\lambda\)-comp}{lambdacomp}\xspace}
\newcommand{\lambdacalcplus}%
{%
  \texorpdfstring%
  {\(\lambda\)-Calculus (+ \(\alpha\))}%
  {Lambda Calculus (+ more)}%
  \xspace%
}
\newcommand{\JJ}{Junyoung~Jang}
\newcommand{\JJemail}{{\scriptsize junyoung.jang@mail.mcgill.ca}}
\newcommand{\JJuniv}{{\scriptsize McGill University}}

\title%
[\lambdacomp: A Primitive \lambdacalcplus Compiler Based on CBPV]%
{\lambdacomp}%
\subtitle{A Primitive \lambdacalcplus Compiler Based on CBPV}%
\author[\JJ]{%
  \texorpdfstring%
  {\parbox{10em}{\centering \JJ\\\JJemail\\\JJuniv}}%
  {\JJ}%
}%
\date{}%

% LTeX: enabled=true

\begin{document}
\begin{frame}[plain]
  \titlepage
\end{frame}

\begin{frame}{Motivation}
  \begin{tikzpicture}[fulloverlay]
    \node at (0.5, 0.8) {Why \textcolor<2-3>{orange}{compiling} \textcolor<4->{orange}{\acf{CBPV}?}};

    \node[visible on=<3>,text width=\textwidth,align=center] at (0.5, 0.5)
    {%
      To understand how to compile modal \(\lambda\)-calculus\\
      (as an experience)%
    };

    \node[visible on=<5->] at (0.5, 0.55)
    {Being closer to low-level (value-computation distinction)};

    \node[visible on=<6->] at (0.5, 0.45)
    {Allowing simple yet useful optimizations};
  \end{tikzpicture}
\end{frame}

\begin{frame}[fragile]{\acf{CBPV}}
  \begin{tikzpicture}[fulloverlay]
    \node[visible on=<2->,alt=<8->{gray}{}] (vttitle) at (0.3, 0.8)
    {Value Types\qquad\(A,B,\ldots\)};

    \node[visible on=<3->,alt=<4-7>{gray}{}] (cttitle) at (0.7, 0.8)
    {Computation Types\qquad\(S,T,\ldots\)};

    \matrix[visible on=<4->,alt=<8->{gray}{},align=center,matrix of nodes,%
    row 1/.style={visible on=<5->,alt=<5>{orange}{}},%
    row 2/.style={visible on=<6->,alt=<6>{orange}{}},%
    row 3/.style={visible on=<7->,alt=<7>{orange}{}}] (vtdesc) at (0.3, 0.65)
    {%
      \tyInt\\
      \tyBool\\
      \tyUp{S}\\
    };

    \matrix[visible on=<8->,align=center,matrix of nodes,%
    row 1/.style={visible on=<9->,alt=<9>{orange}{}},%
    row 2/.style={visible on=<10->,alt=<10>{orange}{}},%
    row 3/.style={visible on=<11->,alt=<11>{orange}{}}] (ctdesc) at (0.7, 0.65)
    {%
      \tyDown{A}\\
      \tyFun{A}{S}\\
    };

    % \node[visible on=<2->,alt=<11-16>{gray}{},alt=<17-18>{orange}{}] (vtitle) at (0.3, 0.5)
    % {Values \qquad\(U,V,W,\ldots\)};

    % \node[visible on=<3->,alt=<4-9>{gray}{},alt=<10>{orange}{}] (ctitle) at (0.7, 0.5)
    % {Computations\qquad\(L,M,N,\ldots\)};

    % \matrix[visible on=<4->,alt=<11->{gray}{},align=center,matrix of nodes,%
    % row 1/.style={visible on=<5->,alt=<5>{orange}{}},%
    % row 2/.style={visible on=<6->,alt=<6>{orange}{}},%
    % row 3/.style={visible on=<7->,alt=<7>{orange}{}},%
    % row 4/.style={visible on=<8->,alt=<8>{orange}{}},%
    % row 5/.style={visible on=<9->,alt=<9-10>{orange}{},alt=<12>{orange}{}}] (vdesc) at (0.3, 0.3)
    % {%
    %   \(x\) & {\small (Variable)}\\
    %   \(i\) & {\small (Integer)}\\
    %   \tmTrue & \\
    %   \tmFalse & \\
    %   \tmThunk{L} & \\
    % };

    % \draw[visible on=<10>,<-,orange] (vdesc-4-1) -| ($(ctitle.west) - (0.05, 0)$) -- (ctitle);

    % \matrix[visible on=<11->,align=center,matrix of nodes,%
    % row 1/.style={visible on=<12->,alt=<12>{orange}{},alt=<18>{orange}{}},%
    % row 2/.style={visible on=<13->,alt=<13>{orange}{},alt=<18-19>{orange}{}},%
    % row 3/.style={visible on=<14->,alt=<14>{orange}{}},%
    % row 4/.style={visible on=<15->,alt=<15>{orange}{}},%
    % row 5/.style={visible on=<16->,alt=<16>{orange}{},alt=<18-19>{orange}{}},%
    % row 6/.style={visible on=<17->,alt=<17>{orange}{}}] (cdesc) at (0.7, 0.3)
    % {%
    %   \tmForce{U}\\
    %   \tmReturn{U}\\
    %   \tmTo{L}{x}{M}\\
    %   \tmLam{x}{L}\\
    %   \tmApp{L}{U}\\
    %   \tmPrintInt{U}{L}\\
    % };

    % \draw[visible on=<12>,<->,orange] (vdesc-5-1) -| (cdesc-1-1);

    % \begin{scope}[every path/.style={visible on=<18-19>,<-,orange}]
    %   \draw[visible on=<18>] (cdesc-1-1) -| ($(vtitle.east) + (0.05, 0)$) -- (vtitle);
    %   \draw (cdesc-2-1) -| ($(vtitle.east) + (0.05, 0)$) -- (vtitle);
    %   \draw (cdesc-5-1) -| ($(vtitle.east) + (0.05, 0)$) -- (vtitle);
    % \end{scope}
  \end{tikzpicture}
\end{frame}

\begin{frame}{Usual \(\lambda\)-Calculus Compile}
  \begin{tikzpicture}[fulloverlay]
    \matrix[matrix of nodes,row sep={0.15\paperheight,between origins},nodes={draw,rectangle},%
    row 1/.style={alt=<2->{gray}{}},%
    row 5/.style={alt=<2->{gray}{}}] (steps) at (0.5, 0.5)
    {
      \strut Elaboration/Type Checking\\
      \strut Arity Analysis\\
      \strut Closure Conversion\\
      \strut Optimizations\\
      \strut Code Generation\\
    };
    \foreach \a in {1,4}%
    {%
      \pgfmathtruncatemacro\b{\a+1}
      \draw[alt=<2->{gray}{},->] (steps-\a-1) -- (steps-\b-1);%
    }
    \foreach \a in {2,3}%
    {%
      \pgfmathtruncatemacro\b{\a+1}
      \draw[->] (steps-\a-1) -- (steps-\b-1);%
    }
  \end{tikzpicture}
\end{frame}

\begin{frame}{\(\lambda\)-Abstraction as Computation}
\end{frame}

\begin{frame}{Where has the closure gone?}
\end{frame}

\begin{frame}{Arity Analysis for CBPV}
\end{frame}

\begin{frame}{Full \(\lambda\)-Lifting}
\end{frame}

\begin{frame}{\(\eta\)-Reducing Optimization}
\end{frame}

\begin{frame}{Effect-sensitive Optimizations}
\end{frame}
\end{document}
